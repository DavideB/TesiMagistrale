
\addcontentsline{toc}{chapter}{\prefacename}
\pdfbookmark[1]{Preface}{Preface}

\chapter*{Introduction}
Cloud computing has become a widely used form of service oriented computing, where
infrastructure and solutions are offered as a service. Cloud has dramatically changed the way computing infrastructures are abstracted and used. Some of the most intriguing features of cloud computing are elasticity (e.g. on demand resource scaling), pay-per-use, no upfront capital investment, low time to market and transfer of risk. The term "big data" is used to describe a large amount of data, that can be structured, like in the traditional relational databases, semi-structured, like in the self-described XML or JSON documents, or unstructured, like in the logfiles collected mostly by web applications to monitor usage or other user's preferences. More properly, we call big data those that cannot be handled using traditional database and software technologies. Today, every second 8,411 Tweets are sent, 902 Instagram photos are uploaded, 1,502 Tumblr posts are created, 3,690 Skype calls are done, 73,116 Google searches are performed and 2,780,000 emails are sent\footnote{Data source:  \url{http://www.internetlivestats.com/}}. This data is collected and analyzed. Gartner’s defines big data as data that contains greater variety arriving in increasing volumes and with ever-higher velocity. This is known as the three V's characterizing  big data: Volume, Velocity,Variety~\cite{WhatIsBigData}[https://www.oracle.com/big-data/guide/what-is-big-data.html]. Volume is important because the amount of data matters. With big data, you’ll have to process high volumes of low-density, unstructured data. This can be data of unknown value, such as Twitter data feeds, clickstreams on a webpage or a mobile app, or sensor-enabled equipment. For some organizations, this might be tens of terabytes of data. For others, it may be hundreds of petabytes. Velocity is the fast rate at which data is received and (perhaps) acted on. Normally, the highest velocity of data streams directly into memory versus being written to disk. Some internet-enabled smart products operate in real time or near real time and will require real-time evaluation and action. Variety refers to the many types of data that are available. Traditional data types were structured and fit neatly in a relational database. With the rise of big data, data comes in new unstructured data types. Unstructured and semistructured data types, such as text, audio, and video require additional preprocessing to derive meaning and support metadata. By means of hardware virtualization, cloud computing services satisfies all the requested requisites needed to manipulate big data. Elasticity and redundacncy provided by cloud computing also enable big data application high availability, scalability and fault tolerance.
Big data also represent an unprecedented business opportunity for many companies which started to deliver big data applications as a service. According to SoftwareTestingHelp\footnote{Data source: \url{https://www.softwaretestinghelp.com/big-data-companies/}}, these are the top 10 big data companies of 2019: IBM\footnote{\url{https://www.ibm.com/}}, HP Enterprise\footnote{\url{https://www.hpe.com/}}, Teradata\footnote{\url{https://www.teradata.com/}}, Oracle\footnote{\url{https://www.oracle.com/}}, SAP\footnote{\url{https://www.sap.com/}}, Dell EMC\footnote{\url{https://www.dellemc.com/}}, Amazon\footnote{\url{https://www.amazon.com/}}, Microsoft\footnote{\url{https://www.microsoft.com/}}, Google\footnote{\url{https://www.google.com/}}, VMware\footnote{\url{https://www.vmware.com/}}.
Big data applications are used to transform, aggregate and analyze a large amount of data in an easy and efficient way. Specialized frameworks are used to transform these applications in atomic parts that can be executed in a distributed cluster of physical or virtual machines. The limit to the level of parallelism we can obtain is given by the number of machines and the amount of synchronization (e.g. aggregations, grouping) needed among the chunks of data representing the intermediate results. This paradigm has been historically represented by the map-reduce programming model firstly introduced by Google\footnote{\url{https://ai.google/research/pubs/pub62}}. Nowadays, more advanced solutions are available, such as Apache Spark\footnote{\url{http://spark.apache.org/}} and Apache  Tez\footnote{\url{http://tez.apache.org/}} that provide a greater flexibility and allow building large-scale data processing applications using a DAG based structure.
One of the most popular cluster computing framework for big data analytics is Apache Spark [Jorge L Reyes-Ortiz, Luca Oneto, and Davide Anguita]~\cite{articleApacheSpark:2015}. [“Big data analytics in the cloud: Spark on hadoop vs mpi/openmp on beowulf.” In: Procedia Computer Science 53 (2015), pp. 121– 130], a recently developed distributed platform from UC Berkley that exploits in-memory computation for easy, at-scale, computation~\cite{bookSpark:2016}[Ilya Ganelin, Kai Sasaki, Ema Orhian, and Brennon York. Spark:Big Data Cluster Computing in Production. John Wiley \& Sons, 2016]. Spark provides a fast and general data processing platform, letting users execute programs 100x faster in memory or 10x faster on disk than Hadoop, indeed in 2014 it won the Daytona GraySort contest as the fastest open source engine for sorting a petabyte~\cite{articleApacheSpark:2016}[Matei Zaharia, Reynold S Xin, Patrick Wendell, Tathagata Das, Michael Armbrust, Ankur Dave, Xiangrui Meng, Josh Rosen,
Shivaram Venkataraman, Michael J Franklin, et al. “Apache Spark: A unified engine for big data processing.” In: Communications of the ACM 59.11 (2016), pp. 56–65]. Spark is fault-tolerant and is designed to run on commodity hardware. It generalizes the two stage Map-Reduce to support arbitrary DAG. The main advantage of Spark with respect to previous cluster computing frameworks is the fast data sharing between operations. For example, Apache Hadoop requires intermediate data to be written to disk in order to be accessible by the following operations, Spark instead allows to execute in-memory computing. Spark offers a quick way of writing code by means of high-level operators provided in the API: Spark Core, Spark SQL, Spark Streaming, MLlib (machine learning), GraphX (graph). Spark integrates well with various storage systems,  including Amazon S3, Hadoop HDFS and any POSIX-compliant file system. Spark provides its own cluster manager, but it can also run on clusters managed by Hadoop Yarn or Apache YARN. Spark is often used for in-memory computation, but is also capable of handling workloads whose size exceeds the aggregate cluster memory. Quality of Service (QoS) notion in big data application differ by application type. Interactive applications are usually assessed according to response time or throughput, and their fulfillment depends on the intensity and variety of the incoming requests. Big data applications might require a single batch computation on a very large dataset, thus QoS must consider the execution of a single run. In this domain QoS is often called deadline, or the desired duration of the computation. Many factors influence the duration of an application execution, surely resource allocation and scheduling greatly influence the duration. We have a resource allocation problem when applications have different structures, run in contexts with different amount of resources or size of input datasets. We have a scheduling problem when many applications run concurrenlty on the same hardware, so that each application cannot have the totality of the resources assigned to itself. Satisfying deadline-based QoS constraints is a problem related to resource allocation, since the amount of allocated resources determines the duration of the execution of Spark applications. The simplest option available on all cluster managers is static partitioning of the resources, in this way each application is given a maximum amount of resources it can use, and holds them for the whole execution. Memory sharing across applications is currently not provided. Spark also provides a mechanism to dynamically adjust the resources assigned to a specific application according to the workload. Applications may give resources back to the cluster if they are no longer used and re-acquire them again when needed. 

The execution of Spark applications is based on the definition of the execution order and parallelism of the different jobs, given data and available resources. Spark keeps track of these dependencies in a graph that we will refer to as the (parallel) execution plan of the application.


 In this thesis work, we investigate the resource allocation problem related to running big data multi-DAG applications with deadline-based QoS constraints. In this field, the state-of-the-art big data framework is Apache Spark. This thesis is based on an extension of Spark, called xSpark, which offers dynamic resource allocation and enforces QoS constraints.
 A previous work done on xSpark have studied the estimation of execution times and addressed the resource allocation problem in order to meet user defined deadlines in Spark, while another work has tackled the resource scheduling problem in order to establish a policy for the management of the deadlines when multiple applications run simultaneously on the same hardware. 
 All the previous works on estimation of the execution times and dynamic provisioning of resources have always assumed the existance of a unique execution plan, given the computing resources at hand. This assumption is a simplification of real-world applications, since the actual execution plan is generally different across different program paths for applications that include conditional branches and loops in their code. Thus, the assumption of a unique execution plan limits the precision of analysis/prediction techniques.

 To respond to the questions raised by the investigation of resource allocation problems related to running big data multi-DAG applications with deadline-based QoS constraints, we propose a solution, \tool, that integrates the outcome of a research work [inserire qui la citazione dell'articolo di GQ/ricerche correlate di GD?] that makes use of an original combination of lightweight symbolic execution and search-based test generation to help identify the proper execution plans dynamically. It uses \dSymb, a novel technique that: i) automatically extracts all possible execution plans of a Spark application along with dedicated launchers with properly synthesized data that can be used for profiling, and ii) tunes the allocation of resources at runtime, based on the knowledge of the execution plans for which the path conditions hold. An initial set of empirical data is provided that support our research hypothesis that \dSymb can effectively complement \cSpark to help predict the execution duration and the dynamic allocation of resources.

A previous work on xSpark addressed the resource allocation problem, in order to meet user defined deadlines in Spark. xSpark is a Spark extension that offers optimized and elastic provisioning of resources. This is obtained by using nested control loops. A centralized loop is implemented on the master node, it controls the execution of different stages of an application. Multiple local loops, one per executor, focus on task execution. xSpark exploits the log data provided by an initial profiling
application execution in order to create an enriched DAG of the application, holding  information about the stages. At runtime, the annotated DAG is used to understand how much work has already been done and how much work still needs to be done. Since all executions of the same application use the same DAG, xSpark requires that the applications do not contain branches or loops, which might be resolved in different ways at runtime. The centralized control loop is activated before the execution of each stages and uses a heuristic to assign a deadline to the stage and calculate the required CPU cores needed to satisfy the stage deadline, using the provided enriched DAG and the user requested application deadline. Many factors can influence the actual performance and invalidate the prediction. Local control loops have the objective to
counteract those imprecisions, by dynamically modifying the amount of CPU cores assigned to the executors during the execution of a stage. A control theory algorithm determines the amount of CPU cores to be allocated to the executor for the next control period. Docker is used in order to tune the number of CPU cores allocate to
the executors, which are run inside lightweight containers\footnote{\url{https://www.docker.com}}. xSpark is able to use less resources than native Spark and can complete executions with less than 1\% error in terms of set deadlines.
The problem of meeting a particular deadline resort to a resource allocation problem (addressed in xSpark previous work) or a scheduling problem, since other applications may be in execution on the same hardware.
The scheduling problem on xSpark prevents executing multiple applications at the same time. This is due to the fact that there is no policy to share resources across applications. This problem has been solved by a the previous works done on xSpark, however the solution proposed by this thesis to the problem of running big data multi-DAG applications with deadline-based QoS constraints is built on a version of xSpark that does not include this capability.

In order to execute big data applications, Spark~\cite{Zaharia2010} divides the computation into different phases and split the input dataset into partitions that are stored in a distributed fashion and processed in parallel. Spark exploits in-memory processing and storage as a means to reuse partial results. Spark applications can be written in Java, Python, or Scala and exploit two types of dedicated operations: \textit{action}s and \textit{transformation}s. 
Actions trigger (distributed) computations that return results to the application. Transformations carry out data transformation in parallel. Spark groups operations into \textit{stage}s and then into \textit{job}s. A stage is composed by a sequence of transformations that do not require data shuffling, while a job identifies a sequence of transformations between two actions. For each job, Spark computes a \textit{Parallel Execution Plan} (\plan) to maximize the parallelism while executing an application. In fact a stage is, by definition, executed in parallel but different stages can also be executed concurrently. For this reason, Spark materializes \plans as directed acyclic graphs of stages, while the complete \plan of an application is simply the sequence of the \plans of its jobs.
All these approaches implicitly assume that the \plan of an application is unique for any possible input dataset and input parameters, but this is at least simplistic for all but the most trivial applications. 
In general, application code embeds branches and loops, and different input data and parameter values may make the application traverse different program paths, and thus materialize a different \plan. For example, a Spark application can evaluate partial results through actions, and then use these results in conditional expressions of program branches.

The solution covered by this thesis proposes the use of an original combination of lightweight symbolic execution and search-based test generation to properly infer the actual \plan, given used data and parameters. Our approach is called \dSymb (\textit{Symbolic Execution-driven Extraction of Parallel Execution Plans})
and relies on lightweight symbolic execution of the application's code to derive a representative set of execution (path) conditions of the \plans in the application. It then uses these conditions with a search-based test generation algorithm to compute sample input datasets to execute each \plan and profile the application. The approach is supported by a prototype tool, also called \dSymb, that identifies the \model of applications, that is, the set of \plans, along with their respective path conditions and profiling data. It also incrementally evaluates the \model against the  concrete values of the symbolic variables, to stay aware of the \plans for which the path conditions continue to hold. This information can then be used to refine 
the actual \plan. 

To properly investigate the benefit of \dSymb, we integrated the tool into a new version of  \cSpark called \tool to understand how the dynamic selection of the best \plan can help in allocating resources more efficiently. A special-purpose component is now in charge of providing the worst-case \plan. At each execution step, the component feeds \cSpark with the actual worst-case \plan, chosen among those that are still feasible, that is, their path condition still holds true. The evaluation shows that i) our approach extracts all the possible \plans generated from different application executions, and  ii) \dSymb helps \cSpark reduce the number of deadline violations and allocate resources more efficiently.

We evaluated  \dSymb by integrating it with \cSpark to control the parallel execution of two example Spark applications. Our evaluation addressed two main research questions:

\begin{enumerate}%[\boldmath$RQ_1$] 
	\item Can \dSymb effectively control the execution of the Spark applications?
	\item To what extend can \dSymb improve the resource allocation capabilities of
	\cSpark, given it used a single, constant \plan?
\end{enumerate}

\section*{Motivation and contribution}
The main reason behind this research and thesis work is to address the problem of executing multi-DAG deadline-constrained big data Spark applications, and give an answer to the research questions, identified in the previous section, inherently associated to this problem. The aim of this work is to give a contribution in terms of knowledge about the application of symbolic execution to the theoretical solution of the problem, and give a contribution in practical terms by providing: i) a modified xSpark platform to enable the runtime management of multi-DAG deadline-constrained big data applications and ii) a toolchain to identify the applications' execution paths,  extract their associated path conditions and generate a path condition evaluator, submit the application and its metadata to the modified xSpark platfom and collect performance data for evaluating the QoS of the execution. The DAG of an xSpark application, which is composed by 

\section*{Results }
The solution was tested with two applications:  \textit{Promocalls}\footnote{https://github.com/seepep/promocalls}, a paradigmatic example  develop at Deib Labs of Politecnico di Milano, and Louvain, a Spark implementation of the \textit{Louvain} algorithm~\cite{Louvain}, that we downloaded from a highly rated GitHub repository\footnote{https://github.com/Sotera/spark-distributed-louvain-modularity}. Louvain exploits \textit{GraphX}, a Spark graph processing library to represent large networks of users and analyze communities in these networks. 

The results of the tests confirm the validity of our claim, that is, that being aware of the different PEPs generated by Spark applications helps analyze and control their performance/execution time.
