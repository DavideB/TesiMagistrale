\chapter{Introduction} \label{chap:Introduction}
%\begin{flushright}{\slshape    
%   Science, my boy, is made up of mistakes, but they are mistakes
%   which it is useful to make, because they lead little by little
%   to the truth}. \\ \medskip --- \citeauthor{verne_journey:1957}
%   \citetitle{verne_journey:1957} \citeyear{verne_journey:1957}
%\end{flushright} 

%\lettrine[lines=4]{\textcolor{purple}{C}}{loud} computing has become a widely used form of service oriented computing, where infrastructure and solutions are offered as a service. The cloud has dramatically changed the way computing infrastructures are abstracted and used. Some of the most intriguing features of cloud computing are elasticity (e.g. on demand resource scaling), pay-per-use, no upfront capital investment, low time to market and transfer of risk. 
\lettrine[lines=4]{\textcolor{purple}{C}}{loud} computing has become a widely used form of service oriented computing, thanks to the provisioning of infrastructures and solutions as a service. Cloud computing has established a totally innovative paradigm of virtualization and use of IT infrastructures. Elasticity (on-demand resource provisioning), metered usage of resources and pay-per-use, no upfront capital investment, fast time-to-market and risk transfer represent most of the most interesting features of cloud computing.

The term "big data" is used to describe a research field that deals with ways to analyze and extract information from data sets containing structured and unstructured data whose size is so large that makes the processing of data with traditional databases and applications very difficult or practically impossible. Data sets can contain a large amount of data that can be structured, like in the traditional relational databases, semi-structured, like in the self-described XML or JSON documents, or unstructured, like in the logfiles collected mostly by web applications to monitor usage or other user's preferences. More properly, big data deals with those data that cannot be handled using traditional database and information technologies. 

\section{Context}\label{sec:context}
Nowadays, every second 2,780,000 emails are exchanged, 3,690 Skype calls are made, 902 Instagram photos are posted, 73,116 Google searches are performed, 1,502 Tumblr posts are created, and 8,411 Tweets are sent~\cite{misc:InternetLiveStats}. This is only a small fraction af all the data which is continuously collected and analyzed.

Gartner~\cite{Gartner} defines big data as data that contains greater variety arriving in increasing volumes and with ever-higher velocity. This is known as the three V's characterizing  big data: Volume, Velocity, Variety~\cite{WhatIsBigData}. 

Volume is important because the amount of data drives both the size of memory infrastructure  and the computational effort for their analysis. Big data requires processing of high volumes of low-density, unstructured data. These can be data of unknown value, such as Twitter data feeds, clickstreams on a webpage or a mobile app, or sensor-enabled equipment. In some cases, this might be tens of terabytes of data, sometimes hundreds of petabytes. 

Velocity is the fast rate at which data is received and acted upon. Normally, highest velocity data streams are directly stored into memory versus being written to disk. Some internet-enabled smart products operate in real time, or near real time, requiring a very fast evaluation and action. 

Variety refers to the many available data types. Traditional data types were structured and are suitable to be stored and managed in a relational database. With the rise of big data, data arrives in an unstructured form. Unstructured and semistructured data types, such as text, audio, and video require an additional preprocessing effort to transform, derive meaning and attach metadata to them. 

The systems used to process big volumes of data need to be scalable, tolerate faults, and provide high availability~\cite{articleBigData:2017}.

%Storing and processing big volumes of data requires scalability, fault tolerance and availability~\cite{articleBigData:2017}. 

Scalability means the ability to maintain a near-linear progression between size of data to process and computational resources to perform the task. A big challenge to scalability is the overhead to keep the numerous data fragments of the intermediate results of the data processing steps in synch. Such overhead could drain so many resources to hinder scalability already above modest scale factors.

Fault tolerance is a technology challenge in big data, especially when processing involves many networked nodes and it becomes cumbersome to retain all the checkpoints/restarts to be  enacted upon partial processing failures. Devising 100\% reliable systems is not an easy task, however the systems can be architected so that the probability of failure falls within the allowed range. 

By means of hardware virtualization, cloud computing services satisfies all the requested requisites needed to manipulate big data. Elasticity and redundancy provided by cloud computing also enable big data application high availability, scalability and fault tolerance.
Big data also represent an unprecedented business opportunity for many companies which started to deliver big data applications as a service. According to SoftwareTestingHelp~\cite{misc:BigDataCompanies}, these are the top 10 big data companies of 2019: IBM~\cite{misc:IBM}, HP Enterprise~\cite{misc:HPE}, Teradata~\cite{misc:Teradata}, Oracle~\cite{misc:Oracle}, SAP~\cite{misc:SAP}, Dell EMC~\cite{misc:EMC}, Amazon~\cite{misc:AWS}, Microsoft~\cite{misc:Microsoft}, Google~\cite{misc:Google}, VMware~\cite{misc:VMware}.

Big data applications are used to transform, aggregate and analyze a large amount of data in an easy and efficient way. These applications are split in atomic parts that can be executed in a distributed cluster of physical or virtual machines by means of specialized frameworks. The limit to the level of parallelism that can be obtain is given by the number of machines and the amount of synchronization needed to keep the data fragments representing the results of intermediate operations synchronized (e.g. aggregations, grouping). This paradigm has been historically represented by the map-reduce programming model firstly introduced by Google~\cite{misc:GoogleMapReduce}. Nowadays, more advanced solutions are available, such as Apache Spark~\cite{misc:ApacheSpark} and Apache Tez~\cite{misc:ApacheTez} that provide a greater flexibility and allow building large-scale data processing applications using a \textit{Directed Acyclic Graph} (DAG) or \textit{Parallel Execution Plan} (\plan) structure to keep track of intermediate results of operations on datasets and determine which parts of the application can be executed in parallel.

%Apache Spark ~\cite{articleApacheSpark:2015} is one of the most popular framework for big data analytics running on clusters of computers in the cloud. 

The Spark processing platform is very popular~\cite{articleApacheSpark:2015} and is distinguished by the generality of the solutions and the speed of data processing and sharing. Compared to Hadoop, applications can be run up to 100x faster when operating in memory and up to 10x when operating on disk. Spark is an award-winning platform as the fastest among the open source engines in data sorting operations~\cite{articleApacheSpark:2016}.

%Spark provides a fast and general data processing platform, letting users execute programs 100x faster in memory or 10x faster on disk than Hadoop, indeed in 2014 it won the Daytona GraySort contest as the fastest open source engine for sorting a petabyte~\cite{articleApacheSpark:2016}. 

Spark embeds fault-tolerance and can run on off-the-shelf hardware. It allows for faster executions by avoiding or limiting the use of the persistent storage and it provides a more sophisticated programming model than the traditional Map-Reduce, based on parallel execution plans (\plans) which are represented as directed acyclic graph (DAG) of phases.
%It generalizes the two stage Map-Reduce to support arbitrary DAG. The main advantage of Spark with respect to previous cluster computing frameworks is the fast data sharing between operations. For example, Apache Hadoop requires intermediate data to be written to disk in order to be accessible by the following operations, Spark instead allows to execute in-memory computing. 

Programming in Spark is made easy by the use high-level functionalities provided by a very rich set of software libraries which compose its API. It embeds specialized libraries for: Spark Core (basic), Spark Streaming, GraphX (graph), MLlib (machine learning), Spark SQL. Spark provides also a seamless integration with POSIX-compliant file system, including HDFS and virtual storage systems, including Amazon S3 and Azure Blob Storage. Spark can be deplyed on Hadoop Yarn or Apache Mesos managed clusters, or can run its own cluster manager.  
 
%Spark provides its own cluster manager, but it can also run on clusters managed by Hadoop Yarn or Apache Mesos. Spark is often used for in-memory computation, but is also capable of handling workloads whose size exceeds the aggregate cluster memory.
%Spark provides its own cluster manager, but it can also run on clusters managed by Hadoop Yarn or Apache Mesos. Spark is often used for in-memory computation, but is also capable of handling workloads whose size exceeds the aggregate cluster memory. 

In order to execute big data applications, Spark~\cite{Zaharia2010} divides the computation into different phases and split the input dataset into partitions that are stored in a distributed fashion and processed in parallel. Spark provides the API in to support applications written in Java, Python or Scala. It stores in memory the partial results of the operations in order to facilitate its reuse, avoiding recalculations of data sets. 
The operations performed by Spark on the data are divided into \textit{action}s and \textit{transformation}s. The \textit{action}s return the results of the distributed calculations to the application, while the \textit{transformation}s perform the transformation of the data in parallel. 

%Spark exploits in-memory processing and storage as a means to reuse partial results. Spark applications can be written in Java, Python, or Scala and exploit two types of dedicated operations: \textit{action}s and \textit{transformation}s. Actions trigger (distributed) computations that return results to the application. Transformations carry out data transformation in parallel. 

Spark groups operations into \textit{stage}s and then into \textit{job}s. A stage is composed by a sequence of transformations that do not require data shuffling, while a job identifies a sequence of transformations between two actions. For each job, Spark computes a \textit{Parallel Execution Plan} (\plan) to maximize the parallelism while executing an application. In fact a stage is, by definition, executed in parallel but different stages can also be executed concurrently. For this reason, Spark materializes \plans as directed acyclic graphs of stages, while the complete \plan of an application is simply the sequence of the \plans of its jobs. 

%The execution of Spark applications is based on the definition of the execution order and parallelism of the different jobs, given data and available resources. Spark keeps track of these dependencies in a graph that we will refer to as the (parallel) execution plan of the application, also called \plan.

A very important measure related to IT applications is the \textit{Quality of Service} (\qos in the reminder of this document).

The notion of \qos in big data application differ by application type. Interactive applications are usually assessed according to response time or throughput, and their fulfillment depends on the intensity and variety of the incoming requests. Big data applications might require a single batch computation on a very large dataset, thus \qos must consider the execution of a single run. In this domain \qos is often called deadline, or the desired duration of the computation. Many factors influence the duration of an application execution, surely resource allocation and scheduling greatly influence the duration. 

A resource allocation problem arises when applications with different structures run in contexts with different amount of resources or size of input datasets. A scheduling problem arises when many applications run concurrenlty on the same hardware, so that each application cannot have the totality of the resources assigned to itself. Satisfying deadline-based \qos constraints is a problem related to resource allocation, since the amount of allocated resources determines the duration of the execution of Spark applications. The simplest option available on all cluster managers is static partitioning of the resources. This way, each application is given a maximum amount of resources it can use, and holds them for the whole execution time. Memory sharing across applications is currently not provided. Spark also provides a mechanism to dynamically adjust the resources assigned to a specific application according to the workload. Applications may give resources back to the cluster if they are no longer used and re-acquire them again when needed.

xSpark, developed at Politecnico di Milano, is an extension of Spark that offers fine-grained dynamic resource allocation using lightweight containers and enforces \qos constraints. xSpark estimates the execution times and allocate resource in order to meet user defined deadlines. A previous work on xSpark has addressed the scheduling problem and established a policy for managing the deadlines when multiple applications run simultaneously on the same hardware.

While we have mentioned availability, fault tolerance and availability as fundamental requirements of big data application, and \qos as a measure of the capability to meet a user-defined deadline when processing big data, in this thesis will focus on \qos.

\section{Problem and Motivation}\label{sec:problem_motivation}
The growing importance of big data applications has favoured the birth of many analysis techniques to estimate the execution time of Spark applications and perform a proper allocation of resources. For example, Islam et al.~\cite{dSpark} propose a solution to statically allocate resources to deadline-constrained Spark applications while minimizing execution costs. Sidhanta et al.~\cite{Sidhanta2016} estimate the duration of Spark applications using a closed-form model based on the size of the input dataset and the size of the available cluster. Alipourfard et. al~\cite{Alipourfard} use Bayesian optimization to generate performance models of Spark applications and compute the best configuration for their execution.  Baresi et al.~\cite{xsparkreport, Quattrocchi2018} propose \cSpark, an extension of Spark that exploits control theory and containers\footnote{\url{https://www.docker.com}.} to scale allocated resources elastically given the execution times of interest and the other applications  on the same cluster.

All the previous works regarding the estimation of the execution times and dynamic provisioning of resources have always assumed the uniquess of the execution plan for an application, given the computing resources available. This assumption is a simplification of real-world applications, since the actual execution plan is generally different across different program paths when the application code includes conditional branches and loops. Hence, the assumption of a unique \plan limits the precision of the analysis and prediction techniques.

xSpark offers optimized and elastic provisioning of resources in order to meet user defined deadlines. This is obtained by using nested control loops. A centralized loop is implemented on the master node and controls the execution of different stages of an application. Multiple local loops, one per executor, focus on task execution. xSpark exploits metadata provided by an initial profiling to create an enriched annotated \textit{Directed Acyclic Graph} (DAG) or \textit{Parallel Execution Plan} (\plan) of the application that holds information about the execution of the stages. At runtime, the annotated \plan is used to understand how much work has already been done and how much work remains to complete the application. Since all executions of the same application use the same \plan, we infer that an xSpark implicit requirement is that applications cannot contain branches or loops, which might be resolved in different ways at runtime. 

The xSpark centralized control loop is activated before the execution of each stage and uses a heuristic to assign a stage deadline and calculate the required CPU cores needed to satisfy that deadline, using the information contained in the enriched \plan and the user-provided  application deadline. Many factors can influence the actual performance and invalidate the prediction. Local control loops have the objective to counteract those imprecisions, by dynamically modifying the amount of CPU cores assigned to the executors during the execution of a stage. A control theory algorithm determines the amount of CPU cores to be allocated to the executor for the next control period. Docker is used to tune the number of CPU cores allocate to the executors, which are run inside lightweight containers~\cite{misc:Docker}. xSpark is able to use less resources than native Spark and can complete executions with less than 1\% error in terms of set deadlines.

%The capability to meet a particular deadline resorts to resource allocation problems (different resources required by different applications) or scheduling problems (more than one application running concurrently on the same hardware). Both the class of problems have been addressed by previous works on xSpark. 

As mentioned earlier in this document, all these examples are based on the assumption that the application \plan is unique for each set of admissible input data and parameter, but this assumption is valid only for the most trivial applications. In fact, the application code can contain branches and loops that can be accessed in different ways depending on the input data and application parameters, and therefore generate different program flows and generate a different \plan. As an example we can take the case of a Spark program that evaluates an intermediate result that is then used to determine the outcome of a conditional expressions of a program branch.

Symbolic execution techniques are exhaustively employed in testing of software programs to help identify unsafe inputs that cause the programs to crash. The use of these techniques to deliver a  full coverage of the possible dangerous inputs is severely limited by the exponential growth of the required computational resources as a consequence of \textit{path explosion}, due to the need to explore all the possible execution paths and solve all the corresponding \textit{path conditions} of a complex program that has many iterative loops and/or conditional branches in its code. As a consequence, we can rapidly resort to unsolvability. The work of this thesis relies on a lightweight symbolic execution  approach~\cite{Baresi-Quattrocchi-Denaro:2019} that values the solvability of symbolic evaluation-based tools on an \textit{efficiency} base instead of a \textit{soundness and completeness} -- often practically unfeasible -- base.

In this thesis work we investigate the resource allocation problem when running big data multi-\plan applications with deadline-based \qos constraints. 
%The outcome of the research should give an answer to the  open questions of how to run multi-DAG, deadline-constrained Spark applications obeying the set deadaline, in an efficient way.
%rimosse le seguenti RQ in questo punto perchè la soluzione non è ancora nota.
%\begin{enumerate}[\textit{Preliminary}\boldmath$RQ_1 : $] 
%	\item Does the solution provide an effective (w.r.t. obeying deadlines) runtime control of the Spark applications?
%	\item Does the solution provide an efficient (w.r.t. resource usage) runtime management of the application?
%\end{enumerate}

The main reason behind this research and thesis work is to address the problem of running multi-\plan, deadline-constrained Spark applications obeying the set deadline, in an efficient way.
%executing multi-DAG deadline-constrained big data Spark applications, and give an answer to the above questions inherently associated to the identified problem.

\section{Solution and Contribution}\label{sec:solution_contribution}
To address the questions raised by the investigation of resource allocation problems related to running big data multi-\plan applications with deadline-based \qos constraints, we propose a solution that leverages \textit{symbolic execution}. 

%alternative text
%Our proposed solution integrates the use of an original combination of lightweight symbolic execution and search-based test generation to help identify the proper execution plans dynamically. It uses \dSymb, a novel technique that: 

%i) automatically extracts all possible execution plans of a Spark application along with dedicated launcher programs and properly synthesized sets of input data that can be used for profiling, and 

%ii) tunes the allocation of resources at runtime, based on the knowledge of the execution plans for which the path conditions hold. 

%An initial set of empirical data is provided that support our research hypothesis that \dSymb can effectively complement \cSpark to help predict the execution duration and the dynamic allocation of resources.

The solution covered by this thesis involves the combined use of a light symbolic execution procedure and search-based test generation to deduce the \plan corresponding to a specified set of data and input parameters~\cite{Baresi-Quattrocchi-Denaro:2019}. The proposed approach is called \dSymb (\textit{Symbolic Execution-driven Extraction of Parallel Execution Plans}).  It leverages a lightweight symbolic execution of the program to extract a set of execution (path) conditions which is representative of the \plans in the application. A search-based test generation algorithm is then used in combination with these set of conditions to create sample datasets that are used to execute each \plan and to profile the application. A prototype tool, also called \dSymb, supports this methodology by identifying the \model of the applications ( e.g. the set of \plans) and their associated path conditions and profiling data. It also builds the \model incrementally by using the concrete values of the symbolic variables to update the \plans whose path conditions are satisfiable. This information can then be used to refine the actual \plan used to concretely execute the application. 

%Our proposed solution is built on a version of xSpark that does NOT support  multiple concurrent applications.
We have integrated the tool into a new version of \cSpark called \tool wuth the aim of systematically test the benefit of \dSymb and to understand how the dynamic selection of the best \plan can help a more efficiently resources allocation. \tool contains a specialized   component that is dedicated to the selection of the worst-case \plan. At each xSpark job boundary, this component injects into the \cSpark scheduler the actual worst-case \plan, which is chosen among the \plans that are still valid (i.e. their path condition still hold). 

In light of the proposed solution, we have identified the followiing research questions:

\begin{enumerate}[\boldmath$RQ_1 : $] 
	\item Can the execution of Spark applications be effectively controlled by \dSymb?
	\item Can the resource allocation capabilities of \cSpark be improved, given it used a single, constant \plan?
\end{enumerate}

 The aim of this work is to give a contribution in terms of knowledge about the application of symbolic execution to the theoretical solution of the identified  problem, and a contribution in practical terms by providing: 
 
 i) a modified xSpark platform to enable the runtime management of multi-\plan deadline-constrained big data applications and 
 
 ii) a toolchain to identify the applications' execution paths,  extract their associated path conditions and generate a path condition evaluator, submit the application and its metadata to the modified xSpark platfom and collect performance data for evaluating the QoS of the execution. 

\section*{Results and Future Works}\label{sec:results_future_works}
The solution was tested with two applications:  \textit{Promocalls}\footnote{https://github.com/seepep/promocalls}, a paradigmatic example  develop at Deib Labs of Politecnico di Milano, and Louvain, a Spark implementation of the \textit{Louvain} algorithm~\cite{Louvain}, that we downloaded from a highly rated GitHub repository\footnote{https://github.com/Sotera/spark-distributed-louvain-modularity}. Louvain exploits \textit{GraphX}, a Spark graph processing library to represent large networks of users and analyze communities in these networks. 

We evaluated  \dSymb by integrating it with \cSpark to control the parallel execution of two example Spark applications. 
%alternative text:
%Our evaluation addressed the two main research questions identified earlier and shows that: 

%i) our approach extracts all the possible \plans generated from different application executions, and  

%ii) \dSymb helps \cSpark reduce the number of deadline violations and allocate resources more efficiently.

The results of the tests confirm the validity of our claim: \tool, being aware of the different PEPs generated by Spark applications, helps analyze and control their performance/execution time, and thus misses fewer deadlines and allocates resources more efficiently than xSpark.

Future works will address the performance improvement of the profiling phase, by using branch-based selection criteria instead of the simple path-based solution adopted in our presented solution. Another direction to explore in a future work is the application of the identified solution to the control of multi-\plan applications with non-strict deadlines.
